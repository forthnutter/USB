% $Id: parapin.tex,v 1.18 2007/07/08 15:44:32 ahooton Exp $

\documentclass{article}
\addtolength{\topmargin}{-.5in}        % repairing LaTeX's huge margins...
\addtolength{\textheight}{1in}        % more margin hacking
\addtolength{\textwidth}{1.5in}
\addtolength{\oddsidemargin}{-0.75in}
\addtolength{\evensidemargin}{-0.75in}

%\usepackage{graphicx,float,alltt,tabularx}
%\usepackage{wrapfig,floatflt}
%\usepackage{amsmath}
%\usepackage{latexsym}
\usepackage{html}


%\floatstyle{ruled}
%\newfloat{Program Fragment}{btp}{lop}
%\newfloat{Program Fragment}{tbp}{lop}



\title{PARAPIN:
A Parallel Port Pin Programming Library for Linux}

\author{Jeremy Elson\\
jelson@circlemud.org\\
and\\
Al Hooton\\
ahooton@users.sf.net}
\date{updated 07 June 2007}

\begin{document}

%%%%%%%%%%%%%%%%%%%%%%%%% Title Page %%%%%%%%%%%%%%%%%%%%%%%%%

\begin{center}
\begin{latexonly}\vspace*{1in}\end{latexonly}
{\Huge PARAPIN:} \\
\vspace{2\baselineskip}
{\huge A Parallel Port Pin Programming Library for Linux}

{\Large Release 1.5.0 \\
\vspace{2\baselineskip}
http://parapin.sourceforge.net/}

\begin{latexonly}\vspace{0.5in}\end{latexonly}
\vspace{\baselineskip}

{\Large Originally authored by \\
Jeremy Elson \\
\begin{latexonly}\vspace{.5\baselineskip}\end{latexonly}

\vfill

This work was supported by DARPA under grant No.\ DABT63-99-1-0011 as
part of the SCADDS project, and was also made possible in part due to
support from Cisco Systems.

\vfill


\begin{latexonly}\vspace{1in}\end{latexonly}
Currently maintained by \\
{\Large Al Hooton \\
\begin{latexonly}\vspace{.5\baselineskip}\end{latexonly}
{\tt ahooton@users.sourceforge.net}}}
\end{center}
\thispagestyle{empty}
\clearpage
%%%%%%%%%%%%%%%%%%%%%%%%%%%%%%%%%%%%%%%%%%%%%%%%%%%%%%%%%%%%%%%%%%%%%%%%%

\begin{latexonly}
\pagenumbering{roman}

\tableofcontents
\clearpage
\end{latexonly}

% This resets the page counter to 1
\pagenumbering{arabic}

\section{Introduction}


{\tt parapin} makes it easy to write C code under Linux that controls
individual pins on a PC parallel port.  This kind of control is very
useful for electronics projects that use the PC's parallel port as a
generic digital I/O interface.  Parapin goes to great lengths to
insulate the programmer from the somewhat complex parallel port
programming interface provided by the PC hardware, making it easy to
use the parallel port for digital I/O.  By the same token, this
abstraction also makes Parapin less useful in applications that need
to actually use the parallel port as a {\em parallel port} (e.g., for
talking to a printer).

Parapin provides a simple interface that lets programs use pins of the
PC parallel port as digital inputs or outputs.  Using this interface,
it is easy to assert high or low TTL logic values on output pins or
poll the state of input pins.  Some pins are bidirectional---that is,
they can be switched between input and output modes on the fly.

Parapin has two \"personalities\": it can either be used as a
user-space C library, or linked as part of a Linux kernel module.  The
user and kernel personalities were both written with efficiency in
mind, so that Parapin can be used in time-sensitive applications.
Using Parapin should be very nearly as fast as writing directly to the
parallel port registers manually.  The kernel module personality 
of parapin is not directly accesible by programs running in userspace.
However, 
a device driver called {\tt parapindriver} is also included that allows
userspace programs to control the parallel port through standard device
system calls such as open(), close(), and ioctl().

Starting with version 1.5.0, a standardized approach to support
multiple language bindings is in place.  The first two bindings to
be released are a Python binding, created by parapin developer
Pedro Werneck (pwerneck at users dot sf dot net), and a java binding
created by parapin developer Neil Stockbridge
(unclouded at users dot sf dot net).
Documentation and examples for a language binding are available in
the subdirectory for that binding, i.e., the parapin\_java directory
holds everything you need to understand and use the java binding to
parapin.

The language bindings infrastructure is quite simple, but there are
established patterns that should be followed to allow consistent
build and usage across all the bindings.  Study
how the two existing bindings have been done, and ask on the
parapin-internals mailing list for assistance if you would like to
work on a binding for another language.

    
Parapin was written by Jeremy
  Elson
while at the University of Southern California's Information Sciences
Institute.  This work was supported by DARPA under grant No.\ 
DABT63-99-1-0011 as part of the SCADDS project, and was also made
possible in part due to support from Cisco Systems.  It is freely
available under the GNU Library General Public License (LGPL).

The device driver {\tt parapindriver} was written by
\htmladdnormallink{Al Hooton}{mailto:ahooton@users.sourceforge.net} ({\tt ahooton@users.sourceforge.net}), current maintainer of the parapin package, and is
also licensed under the GNU Library GeneralPublic License (LGPL).
This work was supported by a very understanding spouse and quite a lot
of black pekoe tea.F

Up-to-date information about Parapin, including the latest version of
the software, can be found at the \htmladdnormallinkfoot{Parapin Home
  Page}{http://parapin.sourceforge.net/}.

{\Huge Warning:}

{\em Attaching custom electronics to your PC using the parallel port
as a digital I/O interface can damage both the PC and the electronics
if you make a mistake. {\bf If you're using high voltage electronics,
a mistake can also cause serious personal injury.  Be careful.}}

If possible, use a parallel port that is on an expansion card, instead of
one integrated onto the motherboard, to minimize the expense of
replacing a parallel port controller that you destroy.  Replacing an 
expansion card is much less expensive than replacing a motherboard.

{\em {\bf USE THIS SOFTWARE AT YOUR OWN RISK.  PARAPIN IS PROVIDED WITHOUT ANY WARRANTY AT ALL.  YOU ASSUME ALL RISK IN USING IT.}}


\section{Parallel Port Specifications}
\label{specs}

This section will briefly outline some of the basic electrical
operating characteristics of a PC parallel port.  More detail can be
found in the \htmladdnormallinkfoot{IBM Parallel Port
FAQ}{http://www.lvr.com/files/ibmlpt.txt}, written by
Zhahai Stewart, or the
\htmladdnormallinkfoot{PC Parallel Port
Mini-FAQ}{http://www.psionics.demon.co.uk/mp3/faqs/lpt-kris.html}
by Kris Heidenstrom.  Those
documents have a lot of detail about registers, bit numbers, and
inverted logic---those topics won't be discussed here, because the
entire point of Parapin is to let programmers control the port {\em
without} knowing those gory details.

The PC parallel port usually consists of 25 pins in a DB-25 connector.
These pins can interface to the TTL logic of an external device,
either as inputs or outputs.  Some pins can be used as inputs only,
while some can be switched in software, on-the-fly, between input mode
and output mode.  Note, however, that it can be dangerous for two
devices to assert an output value on the same line at the same time.
Devices that are using bidirectional pins must agree (somehow) on who
is supposed to control the line at any given time.

From the point of view of Parapin, the pinout of the parallel port is
as follows:

\begin{center}
\begin{tabular}{|c|c|}
\hline
Pin  &  Direction \\
\hline
1 & In/Out \\
\hline
2-9 & In/Out (see note) \\
\hline
10 & Input, Interrupt Generator \\
\hline
11 & Input \\
\hline
12 & Input \\
\hline
13 & Input \\
\hline
14 & In/Out \\
\hline
15 & Input \\
\hline
16 & In/Out \\
\hline
17 & In/Out \\
\hline
18-25 & Ground \\
\hline
\end{tabular}
\end{center}

Pins 2-9---called the parallel port's ``Data Pins''---are {\em
ganged}.  That is, their directions are not individually controllable;
they must be either all inputs or all outputs.  (The actual {\em
values} of the pins---on or off---{\em are} individually
controllable.)  Also, some of the oldest parallel ports do not support
switching between inputs and outputs on pins 2-9 at all, so pins 2-9
are always outputs.  Many PC motherboards allow the user to select the
personality of the parallel port in the BIOS.  If you need to use pins
2-9 as bidirectional or input pins, make sure your port is configured
as a ``PS/2'' port or one of the other advanced port types; ports in
SPP (Standard Parallel Port) mode may not support direction switching
on pins 2-9.

Pin 10 is special because it can generate interrupts.  Interrupt
handling is discussed in more detail in Section~\ref{interrupts}.

Output pins can assert either a TTL high value (between +2.4v and
+5.0v), or a TTL low (between 0v and +0.8v).  The port can not source
much current.  Specs for different implementations vary somewhat, but
a safe assumption (staying within spec) is that the voltage will be at
least 2.5v when drawing up to 2.5mA.  In reality you can sometimes get
away with using an output pin to power a component that uses 3v or
even 5v logic, but the supplied voltage may sag if more than 2.5mA is
drawn.  Input pins are typically spec'ed to be able to sink up to
about 20mA.  For a more detailed treatment of the parallel port
electronics, see the references above.

\section{Parapin Basics}
\label{basics}
\subsection{Userspace version and kernel version}

Parapin has two personalities: one for plain user-space operation
(i.e., linking Parapin with a regular C program), and one for use with
Linux kernel modules ({\tt kparapin}.
The two personalities differ in how they are
compiled (Section~\ref{compiling}) and initialized
(Section~\ref{initialization}).  Also, only the kernel version is
capable of servicing parallel-port interrupts that are generated from
Pin 10.  Otherwise, the two personalities are the same.

The user-space library version of Parapin works very much like any
other C library.  However, kernel programming differs from user-space
programming in many important ways; these details are far beyond the
scope of this document.  The best reference available is
a fantastic book by Rubini and Corbet, \htmladdnormallinkfoot{Linux
Device Drivers}{http://www.ora.com/catalog/linuxdrive3/}.  Go buy it
from your favorite technical bookstore, or download it as an Open Book
from 
\htmladdnormallink{http://www.xml.com/ldd/chapter/book/}{http://www.xml.com/ldd/chapter/book/}
A far inferior alternative (but better than nothing) is the free
\htmladdnormallinkfoot{Linux Kernel Hacker's
Guide}{http://en.tldp.org/LDP/khg/HyperNews/get/khg.html}.

\subsection{Device driver interface}

If you would like to use the kernel version of parapin
from a userspace program without writing your own hardware-specific driver,
you can load the {\tt parapindriver} module
after loading kparapin.  This device driver exposes
the functionality of kparapin through a normal character-device interface
(except for interrupt handling).
The primary advantages are that
access to the parallel port functionality of parapin may be controlled
through /dev filesystem permissions, and all interaction is done through
standard device-oriented system calls such as open(), close(), and
ioctl().

\subsection{Parapin operations}

In Parapin, there are five basic operations possible:

\begin{itemize}
\item Initialize the library (Section~\ref{initialization})
\item Put pins into input or output mode (Section~\ref{configuring})
\item Set the state of output pins---i.e., high/set or low/clear
(Section~\ref{setandclear})
\item Poll the state of input pins (Section~\ref{polling})
\item Handle interrupts (Section~\ref{interrupts})
\end{itemize}

Each of these functions will be discussed in later sections.

Most of the functions take a {\em pins} argument---a single integer
that is actually a bitmap representing the set of pins that are being
changed or queried.  Parapin's header file, {\tt parapin.h}, defines a
number of constants of the form ``{\tt LP\_PINnn}'' that applications
should use to specify pins.  The {\tt nn} represents the pin number,
ranging from 01 to 17.  For example, the command
\begin{verbatim}
        set_pin(LP_PIN05);
\end{verbatim}
turns on pin number 5 (assuming that pin is configured as an output
pin; the exact semantics of {\tt set\_pin} are discussed in later
sections).  C's logical-or operator can be used to combine
multiple pins into a single argument, so
\begin{verbatim}
        set_pin(LP_PIN02 | LP_PIN14 | LP_PIN17);
\end{verbatim}
turns on pins 2, 14, and 17.

Usually, it is most convenient to use {\tt \#define} statements to
give pins logical names that have meaning in the context of your
application.  The documentation of most ICs and other electronics
gives names to I/O pins; using those names makes the most sense.  For
example, a National Semiconductor ADC0831 Analog-to-Digital converter
has four I/O pins called {\tt VCC} (the supply voltage), {\tt CS}
(chip select), {\tt CLK} (clock), and {\tt D0} (data output 0).  A
fragment of code to control this chip might look something like this:
\begin{verbatim}
        #include "parapin.h"

        #define VCC LP_PIN02
        #define CS  LP_PIN03
        #define CLK LP_PIN04
        #define D0  LP_PIN10 /* input pin */
        ...
        clear_pin(CS);  /* pull Chip Select low, tell it to acquire */
        ...
        set_pin(CLK);   /* clock it */
        clear_pin(CLK);
\end{verbatim}

This method has a number of advantages.  First, it makes the code more
readable to someone who has the ADC0831 documentation in hand.
Second, the {\tt \#define} statements summarize the mappings of IC
pins to parallel port pins, making it easier to build the physical
interface.  Also, if the physical interface changes, the {\tt
\#define}s make it easy to remap parallel port pins to new IC pins.

Parapin's header file also provides these constants in an array called
{\tt LP\_PIN}.  This array is useful in some contexts where the pin
number is being specified with a variable instead of a constant.  For
example:
\begin{verbatim}
        /* runway lights: light up pins 1 through 9 in sequence */
        while (1) {
                for (i = 1; i <= 9; i++) {
                        set_pin(LP_PIN[i]);
                        usleep(200);
                        clear_pin(LP_PIN[i]);
                }
        }
\end{verbatim}

Code such as the above fragment would be much more complex if the
programmer were forced to use constants.  There is a direct
correspondence between indices in the array and pin numbers;
accordingly, only indices from 1 to 17 should be used.  Programs
should never reference \verb|LP_PIN[0]| or \verb|LP_PIN[i]| where
\begin{latexonly}$i > 17$.\end{latexonly}
\begin{htmlonly}i > 17.\end{htmlonly}


\section{Compiling and Linking Your Programs to Use Parapin}
\label{compiling}

Note: Please see the file {\tt INSTALL}, located in the root directory
of the parapin distribution, for details on building and installing
parapin on your system.

\subsection{The C Library Version}

The user-space version of Parapin is compiled and linked very much
like any other C library.  If you installed Parapin on your system
using ``make install'', the library ({\tt libparapin.a}) was probably
installed in {\tt /usr/local/lib}.  The header file with the library's
function prototypes and other definitions, {\tt parapin.h}, is
in the location given in the parapin Makefile as
{\tt \{LINUX\_SRC\}/include/}.

Important: {\em Parapin must be compiled with gcc using compiler
optimization of {\tt -O} or better}.  This is because Parapin uses the
{\tt inb} and {\tt outb} functions that exist only with compiler
optimization.

To use the library, first make sure to {\tt \#include} parapin.h in
your C source file.  When linking, add {\tt -lparapin} along with any
other libraries you might be using.


\subsection{The Kernel Module Version}

As with the C library version, kernel source code that uses Parapin
must {\tt \#include} parapin.h.  Make sure that \verb|__KERNEL__| is
defined before the include statement.

Important: {\em Parapin must be compiled with gcc using compiler
optimization of {\tt -O} or better}.  This is because Parapin uses the
{\tt inb} and {\tt outb} functions that exist only with compiler
optimization.

Linking is a little more complicated.  As with any other
interdependent kernel modules, there are actually two ways of using
parapin with your Linux device driver:
\begin{itemize}
\item Compile Parapin independently, insert it into the kernel as its
own module using {\tt modprobe}, and then insert your driver separately
using a different {\tt modprobe} command.  In this case, modprobe and the
kernel resolve the links between your driver and Parapin's functions.
\item Compile Parapin right into your device driver, so that the
entire thing is inserted with a single {\tt modprobe} command.  This can
be less confusing for users if you're distributing your device driver
to other people because it eliminates the extra step of finding,
compiling, and inserting a separate module.  On the other hand, it
will cause conflicts if a user tries to insert two different modules
that use Parapin at the same time.
\end{itemize}

The Makefiles that come with Parapin compiles Parapin into its own
independent module, called {\tt kparapin.o}.  This module can be
inserted into the kernel using {\tt modprobe} (as described in the first
method).  If you'd rather use the second method, and link Parapin
directly with your device driver module instead, do the following:

\begin{enumerate}
\item Copy parapin.c into your driver's source directory.
\item In your driver's Makefile, compile kparapin.o from parapin.c,
{\em with} the \verb|-D__KERNEL__| directive, but {\em without} the
{\tt -DMODULE} directive.
\item Link kparapin.o with all the other object files that make up
your driver.
\end{enumerate}

If your driver only consists of one other source file, and you have
never linked multiple .o's together into a single kernel module, it's
easy.  First, compile each .c file into a .o using {\tt gcc -c} (the
same as with normal C programs that span multiple source files); then,
link all the .o's together into one big .o using {\tt ld -i -o
final-driver.o component1.o component2.o component3.o ...}.

{\bf Important note:} Whichever method you use, Parapin also requires
functions provided by the standard Linux kernel module {\tt parport}.
Make sure you insert the {\tt parport} module with the {\tt modprobe}
command before inserting any
modules that use Parapin; otherwise, you will get unresolved symbols.

\subsection{The device driver 'parapindriver'}

When building a userspace program that will make use of the {\tt parapindriver}
interface to kparapin, you must include {\tt parapindriver.h}.  This header
file defines the device-specific ioctl commands used to communicate
with the device driver.  It also includes {\tt parapin.h}, so your program
can make use of all the {\tt ``LP\_*''} constants.  The {\tt parapindriver}
system calls take arguments using these constants and pass them unchanged
down to the kparapin routines.


\section{Initialization and Shutdown}
\label{initialization}

Before any other Parapin functions can be used, the library must be
{\em initialized}.  Parapin will fail {\em ungracefully} if any
functions are called before its initialization.  This is because the
library's other functions do not check that the library is in
a sane state before doing their jobs.  This was an intentional design
decision because the maximum possible efficiency is required in many
applications that drive the parallel port at high speeds.  The goal is
for Parapin to be almost as fast as directly writing to registers.

The exact initialization method varies, depending on if Parapin
is being used as a C library, as a kernel module, or through the
{\tt parapindriver} device interface.

\subsection{The C Library Version}

C library initialization is performed using the function
\begin{verbatim}
        int pin_init_user(int lp_base);
\end{verbatim}
whose single argument, {\tt lp\_base}, specifies the base I/O address
of the parallel port being controlled.  Common values of {\tt lp\_base}
are 0x378 for LPT1, or 0x278 for LPT2; parapin.h defines the constants
{\tt LPT1} and {\tt LPT2} to these values for convenience.  However,
the exact port address may vary depending on the configuration of your
computer.  If you're unsure, the BIOS status screen displayed while
the computer boots usually shows the base addresses of all detected
parallel ports.

{\bf Programs using Parapin must be running as root when they are
initialized.}  Initialization of the library will fail if the process
is owned by any user other than the super-user because Parapin has to
request the right to write directly to hardware I/O registers using
the {\tt ioperm} function.  The security-conscious programmer is
encouraged to drop root privileges using {\tt setuid} after a
successful call to {\tt pin\_init\_user}.

The return value of {\tt pin\_init\_user} will be 0 if initialization
is successful, or -1 if there is an error.  Applications must not call
other Parapin functions if initialization fails.  Failed
initialization is usually because the program is not running as root.

No shutdown function needs to be called in the C library version of
Parapin.


\subsection{The Kernel Module Version}

Initialization and shutdown in the kernel flavor of Parapin is done
using the following two functions:
\begin{verbatim}
        int pin_init_kernel(int lpt,
                            void (*irq_func)(int, void *, struct pt_regs *));
        void pin_release();
\end{verbatim}

The first function is not as intimidating as it looks.  Its first
argument, {\tt lpt}, is the parallel port number that you want to
control.  The number references the kernel's table of detected
parallel ports; 0 is the first parallel port and is a safe guess as a
default.

The second argument, {\tt irq\_func}, is a pointer to a callback
function to be used for servicing interrupts generated by the parallel
port.  This argument may be NULL if the driver does not need to handle
interrupts.  Details about interrupt handling are discussed in
Section~\ref{interrupts}.

{\tt pin\_init\_kernel} will return 0 on success, or a number less
than 0 on error.  The return value will be a standard {\tt errno}
value such as {\tt -ENODEV}, suitable for passing up to higher layers.
If Parapin initialization fails, the driver {\em must not} call
Parapin's other functions.  As described earlier, this requirement is
not enforced for efficiency reasons.

When a driver is finished controlling the parallel port using Parapin,
it must call {\tt pin\_release}.  The state of the parallel port's
interrupt-enable bit will be restored to the state it was in at the
time {\tt pin\_init\_kernel} was originally called.

Parapin's {\tt pin\_init\_kernel} and {\tt pin\_release} functions
work with the Linux kernel's standard {\tt parport} facility; as noted
above, the {\tt parport} module must be loaded along with any module
that uses Parapin.  When initialized, Parapin will register itself as
a user of the parallel port, and claim {\em exclusive access} to that
port.  This means no other processes will be allowed to use the
parallel port until {\tt pin\_release} is called.

\subsection{Device driver interface}

Before the {\tt parapindriver} device interface may be used, an
entry in /dev must be created that corresponds to the device driver
module loaded in to the kernel.  The easiest way to do this is to
execute the initialization script {\tt ppdrv\_load.sh}, included in
the parapin distribution, late in the normal
boot process with root privileges.
A common way to do this is to run {\tt ppdrv\_load.sh} at the end of
{\tt /etc/rc.local}.  Note that {\tt ppdrv\_load.sh} was
installed in {\tt /usr/local/bin} if you built the {\tt modulesinstall}
target in the distribution Makefile.

The {\tt ppdrv\_load.sh} script performs the following steps, if you
find that you need to do things manually or in some other way:

\begin{enumerate}
\item Loads both parport and kparapin using {\tt insmod(8)}
\item Loads parapindriver, passing the {\tt devname=<device-name>}
argument to the module
\item Determines the major number dynamically assigned to parapindriver
by the kernel, by looking at {\tt /proc/devices} for 
an entry called {\tt <device-name>}
\item Creates {\tt /dev/<device-name>} using {\tt mknod(8)}, giving it the
major number determined above
\item Changes the mode and ownership of {\tt /dev/<device-name>}
\end{enumerate}

These steps only need to be done once each time
the machine is booted.  If there are problems, either {\tt insmod(8)} or
{\tt parapindriver} will display errors on the console and through the
syslog mechanism (if you have it running).

Once parapindriver is successfully loaded, and a corresponding /dev
entry is in place, initialization and shutdown of the parallel port
are easy.  To
initialize the parapin system, just call {\tt open(2)} on
the /dev entry as you would any other device file:

\begin{verbatim}
       int device;
       device = open("/dev/<device-name>", 0);

       if (device < 0) {
         fprintf(stderr, "device open failed\n");
         exit(-1);
       }
\end{verbatim}

Standard system errors with negative values will be returned from
{\tt open(2)} if it fails.  If the device is already open, the return
value will be {\tt -EBUSY}.

To shutdown the parapin system you simply have to call {\tt close(2)}:

\begin{verbatim}
       close(device);
\end{verbatim}

See the example file {\tt ppdrv-test.c} for further information.

\section{Configuring Pins as Inputs or Outputs}
\label{configuring}

As shown in the table in Section~\ref{specs}, most of the pins on
modern parallel ports can be configured as either inputs or outputs.
When Parapin is initialized, it is safest to assume that the state of
all these switchable pins is {\em undefined}.  That is, they must be
explicitly configured as outputs before values are asserted, or
configured as inputs before they are queried.

As mentioned earlier, Pins 2-9 can not be configured independently.
They are always in the same mode: either all input or all output.
Configuring any pin in that range has the effect of configuring all of
them.  The constant {\tt LP\_DATA\_PINS} is an alias that refers to
all the pins in this range (2-9).

The other switchable pins (Pins 1, 14, 16, and 17) may be
independently configured.  Pins 10, 11, 12, 13, and 15 are always
inputs and can not be configured.

It is also worth reiterating that having two devices both try to
assert (output) a value on the same pin at the same time can be
dangerous.  The results are undefined and are often dependent on the
exact hardware implementation of your particular parallel port.  This
situation should be avoided.  The protocol spoken between the PC and
the external device you're controlling to should always agree who is
asserting a value on a pin and who is reading that pin as an input.

\subsection{C library version and Kernel version}

In both the userspace C library version, and the kernel version, 
pins are configured using one of the following three functions:

\begin{verbatim}
        void pin_input_mode(int pins);
        void pin_output_mode(int pins);
        void pin_mode(int pins, int mode);
\end{verbatim}

The {\tt pins} argument of all three functions accepts the {\tt
LP\_PINnn} constants described in Section~\ref{basics}.  The {\tt
pin\_mode} function is just provided for convenience; it does the same
thing as the {\tt pin\_input} and {\tt pin\_output} functions.  Its
{\tt mode} argument takes one of the two constants {\tt LP\_INPUT} or
{\tt LP\_OUTPUT}.  Calling {\tt pin\_mode(pins, LP\_INPUT)} is exactly
the same as calling {\tt pin\_input(pins)}.

Examples:
\begin{verbatim}
        pin_input_mode(LP_PIN01);
        /* Pin 1 is now an input */

        pin_output_mode(LP_PIN14 | LP_PIN16);
        /* Pins 14 and 16 are now outputs */

        pin_mode(LP_PIN02, LP_INPUT);
        /* Pins 2 through 9 are now ALL inputs */

        pin_mode(LP_PIN01 | LP_PIN02, LP_OUTPUT);
        /* Pin 1, and Pins 2-9 are ALL outputs */

        pin_input_mode(LP_DATA_PINS);
        /* The constant LP_DATA_PINS is an alias for Pins 2-9 */
\end{verbatim}

\subsection{Device driver interface}

When using the {\tt parapindriver} device interface, all functionality 
related to pins on the parallel port is invoked via the
{\tt ioctl(2)} system
call.  The ioctl commands for {\tt parapindriver} are defined in the
{\tt parapindriver.h} header file.  The two used to set pins to be
either output pins or input pins are ``PPDRV\_IOC\_PINMODE\_OUT'' and
``PPDRV\_IOC\_PINMODE\_IN'', respectively. 

Examples:
\begin {verbatim}
        ioctl(device, PPDRV_IOC_PINMODE_OUT, LP_PIN01 | LP_PIN02);
        ioctl(device, PPDRV_IOC_PINMODE_IN, LP_PIN11);
\end{verbatim}

Note that the arguments to these ioctl calls use exactly the same
constants as are used by the userspace and kernel versions of parapin itself.
The rules for combining constants using bitwise 'or' are allowed as well.
These constants are made available to your program because the
{\tt parapindriver.h} header file also includes the {\tt parapin.h}
header file to define the constant values.

Return values from these ioctl calls are the same as those defined for
the corresponding functions in {\tt kparapin}.  There is one additional
return value from an ioctl call in to {\tt parapindriver}, the value
{\tt -ENOTTY}.  This indicates that an invalid ioctl command value was
passed down.


\section{Setting Output Pin States}
\label{setandclear}

\subsection{C library version and Kernel version}

Once Parapin has been initialized (Section~\ref{initialization}), and
pins have been configured as output pins (Section~\ref{configuring}),
values can be asserted on those pins using the following functions:
\begin{verbatim}
        void set_pin(int pins);
        void clear_pin(int pins);
        void change_pin(int pins, int state);
\end{verbatim}

The {\tt pins} argument of all three functions accepts the {\tt
LP\_PINnn} constants described in Section~\ref{basics}.  {\tt
set\_pin} turns pins {\em on}, electrically asserting high TTL values.
{\tt clear\_pin} turns pins {\em off}, electrically asserting low TTL
values.  The convenience function {\tt change\_pin} does the same
thing as {\tt set\_pin} and {\tt clear\_pin}; its {\tt state} argument
takes one of the constants {\tt LP\_SET} or {\tt LP\_CLEAR}.
Calling {\tt change\_pin(pins, LP\_SET)} has exactly the same effect
as calling {\tt set\_pin(pins)}.

These three functions will {\em only} have effects on pins that were
previously configured as output pins as described in
Section~\ref{configuring}.  Attempting to assert a value on an input
pin will have no effect.

Note that while the {\em direction} of Pins 2-9 must be the same at
all times (i.e., either all input or all output), the actual {\em
values} of these pins are individually controllable when they are in
output mode.

Examples:
\begin{verbatim}
        pin_output_mode(LP_PIN01|LP_DATA_PINS|LP_PIN14|LP_PIN16|LP_PIN17);
        /* All these pins are now in output mode */

        set_pin(LP_PIN01 | LP_PIN04 | LP_PIN07 | LP_PIN14 | LP_PIN16);
        /* Pins 1, 4, 7, 14, and 16 are all on */

        clear_pin(LP_PIN01 | LP_PIN16);
        /* Pins 1 and 16 are now off */

        change_pin(LP_PIN01 | LP_PIN02, some_integer ? LP_SET : LP_CLEAR);
        /* Pins 1 and 2 are now off if and only if some_integer == 0 */
\end{verbatim}

\subsection{Device driver interface}

Setting pin state through the device driver interface follows all
the same rules as described above.
The ioctl commands used are ``PPDRV\_IOC\_PINSET'' and
``PPDRV\_IOC\_PINCLEAR''.

Examples:
\begin {verbatim}
        ioctl(device, PPDRV_IOC_PINSET, LP_PIN01 | LP_PIN02);
        ioctl(device, PPDRV_IOC_PINCLEAR, LP_PIN01);
\end{verbatim}

Return values from these ioctl calls are the same as those defined for
the corresponding functions in {\tt kparapin}.  There is one additional
return value from an ioctl call in to {\tt parapindriver}, the value
{\tt -ENOTTY}.  This indicates that an invalid ioctl command value was
passed down.


\section{Polling Input Pins}
\label{polling}
\subsection{C library version and Kernel version}

Once Parapin has been initialized (Section~\ref{initialization}), and
pins have been configured as input pins (Section~\ref{configuring}),
the value being asserted by the ``far end'' can be queried using the
following function:
\begin{verbatim}
        int pin_is_set(int pins);
\end{verbatim}

Any number of pins can be queried simultaneously.  The {\tt pins}
argument accepts the {\tt LP\_PINnn} constants described in
Section~\ref{basics}.  The return value is an integer in the same
format.  Any pins that are set (electrically high) will be set in the
return value; pins that are clear (electrically low) will be clear in
the return value.

Pins may only be queried if:
\begin{itemize}
\item They are permanent input pins (Pins 10, 11, 12, 13, and 15), as
described in Section~\ref{specs}, {\bf or}
\item They are bidirectional pins previously configured as input pins,
as described in Section~\ref{configuring}.
\end{itemize}

Any query to an output pin will always return a value indicating that
the pin is clear.  In other words, this function {\em can not} be used
to determine what value was previously asserted to an output pin.

Examples:
\begin{verbatim}
        pin_input_mode(LP_PIN01 | LP_DATA_PINS | LP_PIN17);
        /* Pins 1, 2-9, and 17 are now in input mode, along with
           Pins 10, 11, 12, 13, and 15, which are always inputs */

        /* check the state of pin 1 */
        printf("Pin 1 is %s!\n", pin_is_set(LP_PIN01) ? "on" : "off");

        /* check pins 2, 5, 10, and 17 - demonstrating a multiple query */
        int result = pin_is_set(LP_PIN02 | LP_PIN05 | LP_PIN10 | LP_PIN17);

        if (!result)
                printf("Pins 2, 5, 10 and 17 are all off\n");
        else {
                if (result & LP_PIN02)
                        printf("Pin 2 is high!\n");
                if (result & LP_PIN05)
                        printf("Pin 5 is high!\n");
                if (result & LP_PIN10)
                        printf("Pin 10 is high!\n");
                if (result & LP_PIN17)
                        printf("Pin 17 is high!\n");
        }
\end{verbatim}

\subsection{Device driver interface}

Querying pin state through the device driver interface follows all
the same rules as described above.
The ioctl command used is ``PPDRV\_IOC\_PINGET''.

Examples:
\begin {verbatim}
        int value;
        value = ioctl(device, PPDRV_IOC_PINGET, DATA);
\end{verbatim}

Return values from these ioctl calls are the same as those defined for
the corresponding functions in {\tt kparapin}.  There is one additional
return value from an ioctl call in to {\tt parapindriver}, the value
{\tt -ENOTTY}.  This indicates that an invalid ioctl command value was
passed down.


\section{Interrupt (IRQ) Handling}
\label{interrupts}

The kernel-module version of Parapin lets parallel port drivers catch
interrupts generated by devices connected to the parallel port.  Most
hardware generates interrupts on the rising edge\footnote{Legend has
it that some very old parallel port hardware generates interrupts on
the {\em falling} edge.} of the input to Pin 10.

Before Parapin's interrupt-handling can be used, the Linux kernel
itself must be configured to handle parallel port interrupts.  Unlike
most other hardware devices, the kernel does not detect or claim the
parallel port's interrupts by default.  It is possible to manually
enable kernel IRQ handling for the parallel port by writing the
interrupt number into the special file {\tt /proc/parport/N/irq},
where {\tt N} is the parallel port number.  For example, the following
command tells the kernel that {\tt parport0} is using IRQ 7:
\begin{verbatim}
        echo 7 > /proc/parport/0/irq
\end{verbatim}
If parallel port support is being provided to the kernel through
modules, it is also possible to configure the IRQ number as an
argument to the {\tt parport\_pc} module when it is loaded.  For
example:
\begin{verbatim}
        insmod parport
        insmod parport_pc io=0x378 irq=7
\end{verbatim}
Note that both the {\tt io} and {\tt irq} arguments are required, even
if the parallel port is using the default I/O base address of 0x378.

The actual interrupt number used by the kernel (7 in the examples
above) must, of course, match the interrupt line being used by the
hardware.  The IRQ used by the parallel port hardware is usually
configured in the BIOS setup screen on modern motherboards that have
built-in parallel ports.  Older motherboards or stand-alone ISA cards
usually have jumpers or DIP switches for configuring the interrupt
number.  The typical assignment of interrupts to parallel ports is as
follows:

\begin{center}
\begin{tabular}{|c|c|}
\hline
Port & Interrupt \\
\hline
LPT1 & 7 \\
\hline
LPT2 & 5 \\
\hline
\end{tabular}
\end{center}

These are reasonable defaults if the actual hardware configuration is
not known.

As described in Section~\ref{initialization}, the {\tt
pin\_init\_kernel()} function allows installation of an interrupt
handler function by passing a pointer to the handler as the second
argument.  A {\tt NULL} value for this parameter means that interrupts
are disabled.  A non-NULL value should be a pointer to a callback
function that has the following prototype:

\begin{verbatim}
        void my_interrupt_handler(int irq, void *dev_id, struct pt_regs *regs);
\end{verbatim}

If and only if a pointer to such a handler function is passed to {\tt
pin\_init\_kernel}, the following functions can then be used to turn
interrupts on and off:
\begin{verbatim}
        pin_enable_irq();
        pin_disable_irq();
\end{verbatim}

{\bf Parapin turns parallel port interrupts off by default.}  That is,
no interrupts will be generated until after a call to {\tt
pin\_init\_kernel()} to install the interrupt handler, {\em and} a
subsequent call to {\tt pin\_enable\_irq();}

{\bf Interrupts must not be enabled and disabled from within the
interrupt handler itself.}  In other words, the interrupt handling
function passed to {\tt pin\_init\_kernel()} {\em must not} call {\tt
pin\_enable\_irq()} or {\tt pin\_disable\_irq()}.  However, the
handler function {\em does not} need to be reentrant: the IRQ line
that causes an interrupt handler to run is automatically disabled by
the Linux kernel while the interrupt's handler is running.  There are
other important restrictions on kernel programming in general and
interrupt-handler writing in particular; these issues are beyond the
scope of this document.  For more details, the reader is referred to
the Linux kernel programming guides mentioned in Section~\ref{basics}.

Some PC hardware generates a spurious parallel port interrupt
immediately after the parallel port's interrupts are enabled (perhaps
to help auto-detection of the IRQ number).  Parapin includes a
workaround that prevents this interrupt from being delivered.  This is
done to ensure consistent interrupt behavior across all platforms.


\section{Examples}

Several programs with example code using Parapin are included in
the Parapin distribution.  Look in the ~/examples subfolder.

\section{Limitations and Future Work}

Currently, Parapin only supports a single parallel port at a time.  It
is not possible to have a single instance of the library manage
multiple instances of a parallel port.  This may be a problem for
software that is simultaneously trying to control multiple parallel
ports.  Someday this may be addressed but it will make the interface
messier (a port handle will have to be passed to every function along
with pin specifiers).

The C-library version of Parapin should probably do better probing of
the parallel port, but any desire to do this was limited because it
replicates what is already done by the Linux kernel and we expect that
most production use of parapin will be with the kernel version anyway.

If you have bug reports, patches, suggestions, or any other comments,
please feel free to visit the project page at
\htmladdnormallink{http://parapin.sourceforge.net/}{http://parapin.sourceforge.net/}.
Here you may subscribe to the parapin-users mailing list, and also
submit support requests.

\end{document}
